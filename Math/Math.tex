\documentclass[11pt]{article}
\usepackage{amsfonts}
\usepackage[T1]{fontenc}
\usepackage{mathabx,graphicx}

\newcommand{\test}{\circlearrowright}
\def \loop {\ensuremath{\rotatebox[origin=c]{-90}{$\circlearrowright$}}}
\def \nestedloop {\ensuremath{\rotatebox[origin=c]{-90}{$\circlearrowright$}}^n}

\begin{document}

\section*{Math}





\section{Definition of a line  $\mathbb{L}$}
\subsection{Definition}
\begin{center}
$\mathbb{L} := \{l_i,...$
\\ \vspace{2mm}
$\exists l_i \in \mathbb{L} \hspace{3mm} \forall i$
\end{center}




\section{Whole Numbers $\mathbb{W}$}
\subsection{Definition}
Define the positive number line $\mathbb{W}$, the set of whole numbers
\begin{center}
$\mathbb{W} :=$
\\ \vspace{2mm}
$|S_i| \in \mathbb{W} \hspace{3mm} \forall i$
\end{center}





\section{Injection (one-to-one)}
\subsection{Definition$\lbrack 1 \rbrack$}
\begin{center}
$\hspace{2mm} S_1 = \{a_1,a_2,...,a_n\}; \hspace{2mm} S_2 = \{b_1,b_2,...,b_m\} \hspace{2mm} m \geq n$
\\ \vspace{2mm}
$injective \lbrack S_1, S_2 \rbrack \rightarrow (\forall a_i \in S_1 \hspace{2mm} \exists b_j \in S_2 : f \lbrack a_i \rbrack \rightarrow b_j)$
\\ \vspace{2mm}
$\forall a_i,a_j \in S_1; f \lbrack a_i \rbrack = f \lbrack a_j \rbrack \Longleftrightarrow a_i = a_j \hspace{2mm} \lbrack 2 \rbrack$
\end{center}





\section{Surjection (onto)}
\subsection{Definition $\lbrack 2 \rbrack$}
\begin{center}
$\hspace{2mm} S_1 = \{a_1,a_2,...,a_n\}; \hspace{2mm} S_2 = \{b_1,b_2,...,b_m\} \hspace{2mm} m \geq n$
\\ \vspace{2mm}
$surjective \lbrack S_1, S_2 \rbrack \rightarrow (\forall a_i \in S_1 \hspace{2mm} \exists b_j \in S_2 : f \lbrack a_i \rbrack \rightarrow b_j)$
\end{center}







\section{Bijection (One-to-one and onto)}
Also known as invertible

\subsection{Definition $\lbrack 3 \rbrack$}
\begin{center}
$\hspace{2mm} S_1 = \{a_1,a_2,...,a_n\}; \hspace{2mm} S_2 = \{b_1,b_2,...,b_n\}$
\\ \vspace{2mm}
$Invertible \lbrack S_1,S_2 \rbrack \rightarrow$
\\ \vspace{2mm}
$ (\forall a_i \in S_1 \hspace{2mm} \exists b_j \in S_2 : f \lbrack a_i \rbrack \rightarrow b_j) \land (\forall b_j \in S_2 \hspace{2mm} \exists a_i \in S_1 : g \lbrack b_j \rbrack \rightarrow a_i)$
\end{center}




\section{Hierarchy of Bijections Surjections and Injections}





\section{Vector}
Define vector as a set of more than one cardinalities

\subsection{Definition}
\begin{center}
$\vec{v} :=$
\\ \vspace{2mm}
$|S_1| = x_1; |S_2| = x_2;...;|S_N| = x_N$
\\ \vspace{2mm}
$\vec{v} = \{|S_1|,|S_2|,...,|S_N|\} = <x_1,x_2,...,x_N>$
\end{center}

\subsection{Dimensionality}
\begin{center}
$
dim \lbrack \vec{v} \rbrack = |\vec{v}| :=
$
\\ \vspace{2mm}
$
\vec{v} = \{|S_1|,|S_2|,...,|S_N|\} = <x_1,x_2,...,x_N>
$
\\ \vspace{2mm}
$
| \vec{v} | = N
$
\end{center}



\section{Vector Space}
Define Vector Space $\bold{V}$ as a set of more than one lines
\subsection{Definition}
\begin{center}
$\bold{V} := \{ \mathbb{L}_1,\mathbb{L}_2,...,\mathbb{L}_N\}$
\end{center}



\section{Spans}
A function/system mapping to a vector space

\subsection{Bijective Span}
\subsection{Injective Span}
\subsection{Surjective Span}




\section{Summation Notation}
Define summation $\sum$; the notation for successive additions
\begin{center}
$\sum_{i=1}^{N} a :=$
\\ \vspace{2mm}
$\sum_{i=1}^{N} a = a + \sum_{i=2}^{N} a =$
\\ \vspace{2mm}
$a + a + \sum_{i=3}^{N} a = ... = a + a + ... + a = a * N$
\end{center}




\section{Definition of Series}
Define series
\begin{center}
$series :=$
\\ \vspace{2mm}
$\sum_{i=1}^{N} x_i = x_1 + x_2 + ... + x_N$
\end{center}






\subsection{Discrete Derivative}
Define Derivative for discrete function f[n]; commonly denoted as a difference function
\begin{center}
$
\Delta_n^1 f[n] := f[n+1] - f[n]
$
\end{center}
\vspace{1mm}
We will use the above definition for the remainder of this document \\


\subsubsection{Left Hand Derivative Definition}
\begin{center}
\vspace{1mm}
$
\Delta_{n_l}^1 f[n] = f[n] - f[n-1]
$
\end{center}


\subsection{Zero Order Derivative}
\begin{center}
\vspace{1mm}
$
\Delta_n^0 f[n] := f[n]
$
\end{center}



\subsection{K$^{th}$ Discrete Derivative}
Define the $K^{th}$ derivative of discrete function $f[n]$
\begin{center}
\vspace{1mm}
$
\Delta_n^K f[n] := \Delta_n^{K-1} f[n+1] - \Delta_n^{K-1} f[n]
$
\end{center}




\subsection{K$^{th}$ Discrete Derivative as an Alternating Sum}
\begin{center}
$
\Delta_n^K f[n] := \Delta_n^{K-1} f[n+1] - \Delta_n^{K-1} f[n]
$
\\ \vspace{4mm}
$
= (\Delta_n^{K-2} f[n+2] - \Delta_n^{K-2} f[n+1]) - (\Delta_n^{K-2} f[n+1] - \Delta_n^{K-2} f[n])
$
\\ \vspace{4mm}
$
= (\Delta_n^{K-2} f[n+2] - 2\Delta_n^{K-2} f[n+1] - \Delta_n^{K-2} f[n])
$
\\ \vspace{4mm}
$
=  \sum_{i=0}^K (-1)^{j} \hspace{1mm} (_K C _j) \hspace{1mm} \Delta_n^{0} f[n + j]
$
\\ \vspace{4mm}
$
= \sum_{i=0}^K (-1)^{j} \hspace{1mm} (_K C _j) \hspace{1mm} f[n + j]
$
\end{center}








\subsection{Z Transform}
Define the Z Transform for discrete function f[n]
\begin{center}
$
\mathcal{Z}(f[n]) := \sum_{n=0}^{\infty}f[n]z^{-n}
$
\end{center}





\subsection{Z Transform of 0 Order Derivative}
\begin{center}
\vspace{1mm}
$
\Delta_n^0 f[n] := f[n]
$
\\ \vspace{3mm}
$
\mathcal{Z}(\Delta_n^0 f[n]) = \mathcal{Z}(f[n])
$
\end{center}





\subsection{Z Transform of 1$^{st}$ Derivative}
\begin{center}
\vspace{1mm}
$
\Delta_n^1 f[n] := f[n+1] - f[n]
$
\\ \vspace{3mm}
$
\mathcal{Z}(\Delta_n^1 f[n]) = \mathcal{Z}(f[n+1] - f[n])
$
\\ \vspace{3mm}
$
= \sum_{n=0}^{\infty} (f[n+1] - f[n]) z^{-n}
$
\\ \vspace{3mm}
$
= \sum_{n=0}^{\infty} (f[n+1]z^{-n} - f[n] z^{-n})
$
\\ \vspace{3mm}
$
= \sum_{n=0}^{\infty} f[n+1]z^{-n} - \sum_{n=0}^{\infty} f[n] z^{-n}
$
\\ \vspace{3mm}
$
= \sum_{m=0}^{\infty} f[m+1]z^{-m} - \sum_{n=0}^{\infty} f[n] z^{-n}
$
\end{center}
Let 
\begin{center}
$
\hat{m} = m + 1; \hspace{2mm} m = \hat{m} - 1
$
\\ \vspace{3mm}
$
= \sum_{m=0}^{\infty} f[\hat{m}]z^{-(\hat{m}-1)} - \mathcal{Z}(f[n])
$
\\ \vspace{3mm}
$
= z^1 \sum_{\hat{m}=1}^{\infty} f[\hat{m}]z^{-\hat{m}} - \mathcal{Z}(f[n])
$
\\ \vspace{3mm}
$
= z^1 \sum_{\hat{m}=1}^{\infty} f[\hat{m}]z^{-\hat{m}} + f[0] - f[0] - \mathcal{Z}(f[n])
$
\\ \vspace{3mm}
$
= z^1 \sum_{\hat{m}=0}^{\infty} f[\hat{m}]z^{-\hat{m}} - f[0] - \mathcal{Z}(f[n])
$
\\ \vspace{3mm}
$
= z^1\mathcal{Z}(f[n]) - f[0] - \mathcal{Z}(f[n])
$
\\ \vspace{3mm}
$
\mathcal{Z}(\Delta_n^1 f[n]) = \mathcal{Z}(f[n])(z^1 - 1) - f[0]
$
\end{center}






\subsection{Z Transform of K$^{th}$ Derivative}
\vspace{1mm}
\begin{center}
$
\mathcal{Z}(f[n]) := \sum_{n=0}^{\infty}f[n]z^{-n}
$
\\ \vspace{5mm}
$
\mathcal{Z}(\Delta_n^K f[n]) = \sum_{n=0}^{\infty} \Delta_n^K f[n]z^{-n}
$
\\ \vspace{5mm}
$
= \sum_{n=0}^{\infty} \sum_{i=0}^K (-1)^{j} \hspace{1mm} (_K C _j) \hspace{1mm} f[n + j] z^{-n}
$
\\ \vspace{5mm}
$
= \sum_{n=0}^{\infty} (f[n+K] - (_K C _1) f[n+K-1] +  (_K C _2) f[n+K-2] - ...\pm f[n])z^{-n}
$
\\ \vspace{5mm}
$
= \sum_{n=0}^{\infty} f[n+K]z^{-n} - (_K C _1) f[n+K-1]z^{-n} +  (_K C _2) f[n+K-2]z^{-n} - ...\pm f[n]z^{-n}
$
\\ \vspace{5mm}
$
= z^K \mathcal{Z}(f[n]) +\sum_{i=0}^{K-1} f[i] - (_K C _1)z^{K-1} \mathcal{Z}(f[n]) - \sum_{j=0}^{K-2}f[j] + (_K C _2)z^{K-2} \mathcal{Z}(f[n])+\sum_{k=0}^{K-3}f[k] - ... \pm \mathcal{Z}(f[n])
$
\end{center}
When K is odd
\begin{center}
$
\mathcal{Z}(\Delta_n^K f[n]) = (z-1)^K \mathcal{Z}(f[n]) + \sum_{i=0}^{\frac{n+1}{2}} f[2i] \hspace{3mm} K > 0
$
\end{center}
When K is even
\begin{center}
$
\mathcal{Z}(\Delta_n^K f[n]) = (z-1)^K \mathcal{Z}(f[n]) + \sum_{j=0}^{\frac{n}{2}} f[2j + 1] \hspace{3mm} K > 0
$
\end{center}






% Convergent Functions; Convergence
\section{Convergent Functions}




\subsection{Definition of Converges to}
\begin{center}
\vspace{1mm}
$f[n]$ $converges$ $to$ $C$ = $convergent [f [n],C] = a_o; \hspace{1mm} a_o \in \{\mathbb{T},\mathbb{F}\} =$
\\ \vspace{6mm}
$
|C - f[n+1]| < |C - f[n]| \hspace{2mm} \forall n
$
\\ \vspace{2mm}
$
\land
$
\\ \vspace{2mm}
$
\not \exists K : | C - f[\hat{n}] | > K \hspace{4mm} \forall n; K > 0
$
\end{center}

\subsubsection{Notation}
C is commonly denoted by a limit
\begin{center}
$
C = lim_{n \rightarrow \infty} f[n]
$
\end{center}



\subsection{Definition of General Convergence}
\begin{center}
\vspace{1mm}
$f[n]$ is $convergent$ = $convergent \lbrack f \lbrack n \rbrack \rbrack = a_o; \hspace{1mm} a_o \in \{\mathbb{T},\mathbb{F}\} =$
\\ \vspace{6mm}
$
\exists C :
$
\\ \vspace{2mm}
$convergent[f[n],C] == \mathbb{T}$
\end{center}
Alternatively
\begin{center}
\vspace{1mm}
$f[n]$ is $convergent$ = $convergent \lbrack f \lbrack n \rbrack \rbrack = a_o \in \{\mathbb{T},\mathbb{F}\} =$
\\ \vspace{2mm}
$
\exists C :
$
\\ \vspace{2mm}
$f[n]$ $converges$ $to$ $C$
\end{center}

\subsubsection{Notation}
\begin{center}
\vspace{1mm}
$f[n]$ is $convergent$ = $convergent \lbrack f \lbrack n \rbrack \rbrack = a_o; \hspace{1mm} a_o \in \{\mathbb{T},\mathbb{F}\} =$
\\ \vspace{6mm}
$
\exists lim_{n \rightarrow \infty} f[n]:
$
\\ \vspace{2mm}
$
f[n]$ $converges$ $to$ $lim_{n \rightarrow \infty} f[n]
$
\end{center}






\subsection{Increasing Convergence}
For strictly increasing functions
\begin{center}
$ f \lbrack n \rbrack = \sum_{i=1}^{n} x_i$
\\ \vspace{2mm}
$\forall f \lbrack n \rbrack : convergent \lbrack f \lbrack n \rbrack \rbrack \rightarrow \mathbb{T}$
\\ \vspace{2mm}
$ \lim_{n \rightarrow \infty} f \lbrack n \rbrack = C:=$
\\ \vspace{2mm}
$C >  f[n] \hspace{3mm} \forall n$
\\ \vspace{2mm}
$
f[n+1] > f[n] \hspace{2mm} \forall n
$
\\ \vspace{2mm}
$\not \exists K : C - \sum_{i=1}^{n} x_i > K \hspace{3mm} \forall n; K > 0$
\end{center}
\subsubsection{Prove Increasing Convergence has General Convergence}




\subsection{Decreasing Convergence}
For strictly decreasing functions
\begin{center}
$ f \lbrack n \rbrack = \sum_{i=1}^{n} x_i$
\\ \vspace{2mm}
$\forall f \lbrack n \rbrack : convergent \lbrack f \lbrack n \rbrack \rbrack \rightarrow \mathbb{T}$
\\ \vspace{2mm}
$ \lim_{n \rightarrow \infty} f \lbrack n \rbrack = C:=$
\\ \vspace{2mm}
$f[n] > C \hspace{3mm} \forall n$
\\ \vspace{2mm}
$
f[n+1] < f[n] \hspace{2mm} \forall n
$
\\ \vspace{2mm}
$\not \exists K :\sum_{i=1}^{n} x_i - C > K \hspace{3mm} \forall n; K > 0$
\end{center}
\subsubsection{Prove Decreasing Convergence has General Convergence}




\subsection{Transient Convergence}
For alternating functions
\begin{center}

\end{center}
\subsubsection{Prove Transient Convergence has General Convergence}

\section{Chaotic Convergence}
\begin{center}
\vspace{1mm}
$f[n]$ $is$ $chaotic$ $convergent$ = $chaotic$ $convergent \lbrack f \lbrack n \rbrack \rbrack =$
\\ \vspace{4mm}
$
\exists C,\hat{n} :
$
\\ \vspace{2mm}
$
 |C - f[n]| > |C - f[n+\hat{n}]| \hspace{2mm} \forall n
$
\end{center}
\subsection{Prove General Convergence implies Chaotic Convergence}
\subsection{Prove Choatic Convergence does not necessarily imply General Convergence}




% Definition of Divergent Function
\subsection{Definition of Divergent Function}



\section{Divergence}
\subsection{Definition of Divergence}
\begin{center}
\vspace{1mm}
$
diverges \lbrack f \lbrack n \rbrack \rbrack = \lnot converges[f[n]] = d_o; \hspace{1mm} d_o \in \{\mathbb{T},\mathbb{F}\}
$
\\ \vspace{4mm}
$
:= \not \exists C : convergent[f[n],C] == \mathbb{T}
$
\end{center}





\subsection{Alternate Definition of Divergence}
\begin{center}
\vspace{1mm}
$diverges \lbrack f \lbrack n \rbrack \rbrack = \lnot converges[f[n]] = d_o; \hspace{1mm} d_o \in \{\mathbb{T},\mathbb{F}\}$
\\ \vspace{4mm}
$
= convergent[f[n],C] == \mathbb{F} \hspace{4mm} \forall C
$
\end{center}
\subsubsection{Proof of Equivalence; Alternate Definition of Divergence}





\subsection{Necessary and Sufficient Criteria 1 For Divergence}
Function f[n] diverges if and only if the $K^{th}$ derivative of f[n] is strictly increasing
\vspace{1mm}
\begin{center}
$
diverges \lbrack f \lbrack n \rbrack \rbrack := \not \exists C : convergent[f[n],C] == \mathbb{T}
$
\\ \vspace{4mm}
$
\Longleftrightarrow
$
\\ \vspace{4mm}
$
\Delta_n^{K+1} f[n] > \Delta_n^{K}f[n] \hspace{3mm} \forall K
$
\end{center}
Alternatively
\begin{center}
$
\Delta_n^{K+1} f[n] - \Delta_n^{K}f[n] > 0 \hspace{3mm} \forall K
$
\\ \vspace{4mm}
$
\Delta_{n}^{K+2} > 0 \hspace{2mm} \forall K
$
\end{center}






\subsubsection{Criteria 1; Proof of Necessity and Sufficiency}
\begin{center}
$
diverges \lbrack f \lbrack n \rbrack \rbrack = d_o; \hspace{1mm} d_o \in \{\mathbb{T},\mathbb{F}\}
$
\\ \vspace{4mm}
$
= \not \exists C : convergent[f[n],C] == \mathbb{T}
$
\end{center}
Let
\begin{center}
$
f[n] :
$
\\ \vspace{2mm}
$
\Delta_{n}^{K+2} > 0 \hspace{3mm} \forall K
$
\end{center}




\subsection{Necessary and Sufficient Criteria 2 For Divergence}
\begin{center}
\vspace{2mm}
$
f[n]\hspace{1mm} is \hspace{1mm} Divergent = Divergent[f[n]] = a_o \in \{\mathbb{T},\mathbb{F}\} :=
$
\\ \vspace{4mm}
$
\not \exists c: lim_{n\rightarrow \infty} \Delta_n^K f[n] = c
$
\end{center}








\subsubsection{Criteria 2; Proof of Necessity and Sufficiency}
\begin{center}
$
diverges \lbrack f \lbrack n \rbrack \rbrack = d_o; \hspace{1mm} d_o \in \{\mathbb{T},\mathbb{F}\}
$
\\ \vspace{4mm}
$
= \not \exists C : convergent[f[n],C] == \mathbb{T}
$
\end{center}






\subsection{Verbal Expressions}
\begin{center}
\vspace{1mm}
$
f[n]$ $diverges$ = $f[n]$ $is$ $divergent =
$
\\ \vspace{2mm}
$
f[n]$ $is$ $not$ $convergent= f[n]$ $does$ $not$ $converge
$
\end{center}











%\newpage
%\section*{Appendix}




%\section{Properties of Sums $\sum$}





%\section{For all convergent series, there exists only one bound}
%\begin{center}
%$\forall f \lbrack n \rbrack : convergent \lbrack f \lbrack n \rbrack \rbrack \rightarrow \mathbb{T}$
%\\ \vspace{2mm} 
%$\lim_{n \rightarrow \infty} f \lbrack n \rbrack = C_1; \lim_{n \rightarrow \infty} f \lbrack n \rbrack = C_2$
%\\ \vspace{2mm}
%$\Longleftrightarrow C_1 = C_2$
%\end{center}
%\subsection{Proof}
%Proof by contradiction




\newpage
\section*{Citations}
$
\lbrack 1 \rbrack \hspace{2mm}  https://en.wikipedia.org/wiki/Bijection,\_injection\_and\_surjection\#Injection
$
\\
$
\lbrack 2 \rbrack \hspace{2mm}  https://en.wikipedia.org/wiki/Bijection,\_injection\_and\_surjection\#Surjection
$
\\
$
\lbrack 3 \rbrack \hspace{2mm}  https://en.wikipedia.org/wiki/Bijection,\_injection\_and\_surjection\#Bijection
$
\\
$
\lbrack 4 \rbrack \hspace{2mm} https://www.wolframalpha.com
$

\end{document}