\documentclass[11pt]{article}
\usepackage{amsfonts}
\usepackage[T1]{fontenc}
\usepackage{mathabx,graphicx}

\newcommand{\test}{\circlearrowright}
\def \loop {\ensuremath{\rotatebox[origin=c]{-90}{$\circlearrowright$}}}
\def \nestedloop {\ensuremath{\rotatebox[origin=c]{-90}{$\circlearrowright$}}^n}

\begin{document}


\section*{Set Theory}


\section{Equivalence}
The law of equivalence
\begin{center}
$
a = a
$
\end{center}


\section{Definition of element $\{a\}$}
\begin{center}
$
\{a\} := a
$
\\ \vspace{2mm}
\end{center}




\section{Definition of Empty Set $\{\}$}
\begin{center}
$
\emptyset := \{\}
$
\end{center}





\section{Definition of Cardinality of element a $|a|$}
\begin{center}
$
|\emptyset| := 0
$
\\ \vspace{2mm}
$
|\{a\}| := 1
$
\end{center}





\section{Definition of Union $\cup$}
\begin{center}
$
\emptyset \cup \{a\} := \{a\}
$
\\ \vspace{2mm}
$
\{a_x\}\cup \{a_y\} := \{a_x; a_y\}
$
\end{center}





\subsection{Notation}
Union $\cup$ can be denoted as $,$
\begin{center}
$
\cup = \hspace{.8mm},
$
\end{center}




\subsection{Equivalent Expressions}
\begin{center}
$
\emptyset, a = \emptyset \cup a = \{\} \cup \{a\} = \{\} , \{a\} 
$
\\ \vspace{2mm}
$
a_x,a_y = a_x \cup a_y = \{a_x\} \cup \{a_y\} = \{a_x\},\{a_y\}
$
\end{center}




\section{Definition of Intersection $\cap$}
\begin{center}
$
\emptyset \cap a := \emptyset
$
\\ \vspace{2mm}
$
a_x \cap a_y := \emptyset
$
\\ \vspace{2mm}
$
a_x \cap a_x := a_x
$
\end{center}




\section{Definition of Deletion $\setminus$}
\begin{center}
$
a_x \hspace{2mm}\setminus \hspace{2mm} a_x := \emptyset
$
\end{center}





\section{Definition of Set S}
\begin{center}
$
S := a_1 \cup a_2 \cup ... \cup a_N
$
\\ \vspace{2mm}
$
= \{a_1\} \cup \{a_2\} \cup ... \cup \{a_N\}
$
\\ \vspace{2mm}
$
= \{a_1; \hspace{1mm} a_2; \hspace{1mm} ... \hspace{1mm}; a_N\}
$
\end{center}






\section{Definition of Magnitude of a Set $|S|$}
\begin{center}
$
S := a_1 \cup a_2 \cup ... \cup a_N = \{a_1; \hspace{1mm} a_2; \hspace{1mm} ... \hspace{1mm}; a_N\}
$
\end{center}



\section{Definition of In $\in$}
\section{Definition of For All $\forall$}




\section{Definition of Vector}
\section{Null Property of Vector V}
\begin{center}
$
< \emptyset_i, \emptyset_j, ... > = \emptyset
$
\end{center}
\subsection{Proof}
\begin{center}
$
< \emptyset_i, \emptyset_j, ... > := \{ \emptyset_i, \emptyset_j,...\}
$
\\ \vspace{2mm}
$
\{\emptyset_i\cup \emptyset_j \cup...\}
$
\\ \vspace{2mm}
$
\{\emptyset\}
$
\end{center}




\section{Properties of Empty Set $\emptyset$}
\subsection{$\emptyset_c = \emptyset_d$}
Prove $\emptyset_c = \emptyset_d$
\begin{center}
$
\emptyset_c := \{\}
$
\\ \vspace{2mm}
$
\emptyset_d := \{\}
$
\\ \vspace{2mm}
$
\therefore \emptyset_c = \{\} = \emptyset_d
$
\end{center}
\subsection{$\emptyset \cup \emptyset = \emptyset$}
Prove
\begin{center}
$
\emptyset \cup \emptyset = \emptyset
$
\\ \vspace{6mm}
$
\emptyset := \{\}
$
\\ \vspace{2mm}
$
\emptyset \cup \emptyset = \{\} \cup \{\} := \{\} = \emptyset
$
\\ \vspace{2mm}
$
\therefore \emptyset \cup \emptyset = \emptyset
$
\end{center}






\section{Definition of Universal Set}
Define Universal Set
\begin{center}
$
\Omega := s_i \in \Omega, \forall i \hspace{5mm}
$
\end{center}





\section{Set Union $\cup$}
Define $\cup$ the union of two elements
\begin{center}
$\cup \lbrack l_i,l_j\rbrack = l_i \cup l_j = \{l_i\} \cup \{l_j\} :=$
\\ \vspace{2mm}
$\cup \lbrack l_i,l_j\rbrack \rightarrow \{l_i\} \hspace{2mm} i = j$
\\
$\cup \lbrack l_i,l_j\rbrack \rightarrow \{l_i,l_j\} \hspace{2mm} i \neq j$
\end{center}

\subsection{Translation}
$\cup$ is often read as "and"

\subsection{Comma ,}
In set notation the comma "," denotes union $\cup$
\begin{center}
$l_1 \cup l_2 = \{l_1\} \cup \{l_2\} = \{l_1,l_2\}$
\end{center}





\section{Set Intersection $\cap$}
Define $\cap$, the intersection of two elements
\begin{center}
$l_1 \cap l_2 = \{l_1\} \cap\{ l_2\}$
\\ \vspace{2mm}
$\cap \lbrack l_i,l_j\rbrack \rightarrow \{l_i\} \hspace{2mm} i = j$
\\
$\cap \lbrack l_i,l_j\rbrack \rightarrow \emptyset \hspace{2mm} i \neq j$
\end{center}


\section{Set Subtraction $\setminus$}


\section{Sets}




\subsection{Definition}
Define set $S$ as an ordered union of elements $s_i$

\begin{center}
$
S :=   s_1 \cup  s_2 \cup\ ... \cup s_{n-1} \cup\ s_n = \{s_1,s_2,..s_N\}
$
\end{center}
\subsection{Alternate Notation}
\begin{center}
$
S := s_i \in  S :  i=1,2,...,N-1,N \hspace{5mm}$
\\$
S = \{s_1,s_2,...,s_{N-1},s_N\}
$

\end{center}





\subsection{Magnitude of a Set}
\begin{center}
$
|S| = |\hspace{.5mm} \{x_1,...,x_N\}| = N
$
\end{center}




\subsection{Counting}
\begin{center}
$1,2,...,N = 1:N$
\end{center}



\subsection{Definition Unordered Set}
Set $S$ is unordered if
\begin{center}
$
S = \{x_1,x_2,..x_n\} :=
$
\\
$
x_i,x_j \in S; \hspace{2mm} x_i = x_j; \hspace{3mm}\forall i,j \neq i
$
\end{center}





\subsection{Definition of Unique Set}
\begin{center}
$
a_i, a_j \in S
$
\\
$
a_i \neq a_j \hspace{3mm} \forall i,j \neq i
$
\end{center}





\subsection{Definition of Countable/Uncountable set}
Potentially just a line?
\subsection{Define line $\mathbb{L}$}
Define line $\mathbb{L}$
\begin{center}
$
\mathbb{L} := \{l_0,l_{1},l_{2},...,l_{N-1},l_N,l_{N+1},...
$
\\
$
\Longleftrightarrow \exists l_i \in \mathbb{L} \hspace{2mm} \forall i
$
\end{center}


\section{Hierarchy of Elements to Sets}
Every element is a set, but not all sets are elements







\section{Containment}
\subsection{Contains}
\subsection{Equals = }
Define set equivalence  =
\begin{center}
$
S_1 \subseteq S_2 ; \hspace{3mm} S_2 \subseteq S_1 \hspace{1mm} \Longleftrightarrow S_1 = S_2 \hspace{3mm}
$
\end{center}


\subsection{Subset}



\subsection{Proper Subset Citation}




\subsection{Definition of Complement}
\begin{center}
$
S = \{ s_1,s_2,...,s_N \}
$
\\
$
S^C :=
$
\\
$
s_j: \{ s_j \in \Omega\} \cap \{s_j \not \in S\}; \hspace{2mm} \forall j
$
\end{center}
\subsection{Alternate Notation}
Wikipedia definition of complement
\begin{center}
$
S^C = U - S = \{x \in \Omega : x \not \in S\} \hspace{3mm} 
$
https://en.wikipedia.org/wiki/Complement\_(set\_theory)
\end{center}




\newpage
\section*{Appendix}

\section{Criticism of Union $\cup$}


\section{Alternate Translations}
\subsection{Set}
Collection
\subsection{Magnitude}
Count
\subsection{Intersection}
Mutual Elements
\subsection{Union}
And



\section{Proofs and Properties}
$
1. \vspace{3mm} \hspace{3mm} \Omega \subset \emptyset \\
2. \vspace{3mm} \hspace{3mm} \Omega \cap \Omega = \Omega \\
3. \vspace{3mm} \hspace{3mm} \Omega \cup \Omega = \Omega \\
4. \vspace{3mm} \hspace{3mm} \Omega \cup \emptyset = \Omega \\
5. \vspace{3mm} \hspace{3mm} \Omega \cap \emptyset = \emptyset \\
6. \vspace{3mm} \hspace{3mm} \Omega \cap  S = S \\
7. \vspace{3mm} \hspace{3mm} \Omega \cup S = \Omega \\
8. \vspace{3mm} \hspace{3mm} \emptyset \not\subseteq \Omega \\
9. \vspace{3mm} \hspace{3mm} \emptyset \cup \emptyset = \emptyset \\
10. \vspace{3mm} \hspace{3mm} \emptyset \cap \emptyset = \emptyset \\
11. \vspace{3mm} \hspace{3mm} \emptyset \cap  S = \emptyset \\
12. \vspace{3mm} \hspace{3mm} \emptyset \cup S = S\\
13. \vspace{3mm} \hspace{3mm} \emptyset = \emptyset\\
14. \vspace{3mm} \hspace{3mm}\emptyset = \Omega^C\\
15. \vspace{3mm} \hspace{3mm} \Omega \subseteq  S
$






\end{document}